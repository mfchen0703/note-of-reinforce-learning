\documentclass{article}
\usepackage[UTF8]{ctex}
\usepackage{graphicx}
\usepackage{fontspec}
\usepackage{xcolor}
\usepackage{amsmath}
\usepackage{amssymb}
\begin{document}
\title{Lecture Notes 2}
\author{}
\date{\today}
\maketitle
\section{第二章\ 贝尔曼公式}
\textcolor{blue}{\kaishu*数学直观:return为什么是重要的?}

\textbf{return}可以用来量化并评估某个policy是否是“好的”

v为状态值,r为当前状态的reward,P为与状态空间有关的矩阵,表明不同状态之间的关系

\textbf{状态值}:state value,是在采用某策略的条件下,某个状态出发得到的return的数学期望
\[
v_{\pi}(s) = \mathbb{E}[G_t | S_t = s]
\]
\textcolor{blue}{\kaishu*状态值和return的区别:return是根据某一条trajectory得到的确定的值,而状态值是一个符合概率分布的随机变量}

\textbf{贝尔曼公式的一般形式}:
$$v_{\pi}(s) = \mathbb{E}[G_t | S_t = s]$$
数学推导过程:
\begin{align*}
v_{\pi}(s) &= \mathbb{E}[G_t | S_t = s] \\
&= \mathbb{E}[R_{t+1} + \gamma G_{t+1} | S_t = s] \\
&= \textcolor{blue}{\mathbb{E}[R_{t+1} | S_t = s]} + \gamma \textcolor{blue}{\mathbb{E}[G_{t+1} | S_t = s]}
\end{align*}
\begin{align*}
\textcolor{blue}{\mathbb{E}[R_{t+1} | S_t = s]} &= \sum_{a} \pi(a|s) \mathbb{E}[R_{t+1} | S_t = s, A_t = a] \\
&= \sum_{a} \pi(a|s) \sum_{r} p(r|s, a)r
\end{align*}
\begin{align*}
\textcolor{blue}{\mathbb{E}[G_{t+1} | S_t = s]} &= \sum_{s'} \mathbb{E}[G_{t+1} | S_t = s, S_{t+1} = s'] p(s'|s) \\
&= \sum_{s'} \mathbb{E}[G_{t+1} | S_{t+1} = s'] p(s'|s) \\
&= \sum_{s'} v_{\pi}(s') p(s'|s) \\
&= \sum_{s'} v_{\pi}(s') \sum_{a} p(s'|s, a) \pi(a|s)
\end{align*}
\textbf{总结}:
\begin{align*}
\textcolor{red}{v_{\pi}(s)} &= \mathbb{E}[R_{t+1} | S_t = s] + \gamma \mathbb{E}[G_{t+1} | S_t = s], \\
&= \underbrace{\textcolor{blue}{\sum_{a} \pi(a|s) \sum_{r} p(r|s, a)r}}_{\text{mean of immediate rewards}} + \underbrace{\textcolor{blue}{\gamma \sum_{a} \pi(a|s) \sum_{s'} p(s'|s, a) \textcolor{red}{v_{\pi}(s')}}}_{\text{mean of future rewards}}, \\
&= \textcolor{blue}{\sum_{a} \pi(a|s) \left[ \sum_{r} p(r|s, a)r + \gamma \sum_{s'} p(s'|s, a) \textcolor{red}{v_{\pi}(s')} \right]}, \quad \forall s \in \mathcal{S}.
\end{align*}

贝尔曼公式用于计算某个状态的状态值。描述了不同状态的状态值之间的关系

\textcolor{blue}{\kaishu*贝尔曼公式对状态空间的所有状态都适用}

\textcolor{blue}{\kaishu*通过求解状态值,可以用来评估policy(即公式中的$\pi$})

\textcolor{blue}{\kaishu*公式中的$p(r|s, a)$和$p(s'|s, a)$代表dynamic model(或environment model)}

\textcolor{blue}{\kaishu*通过对比不同策略下同一个状态的状态值,就可以知道相对而言策略的好坏}

\end{document}