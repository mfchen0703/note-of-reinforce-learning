\documentclass{article}
\usepackage[UTF8]{ctex}
\usepackage{graphicx}
\usepackage{fontspec}
\usepackage{xcolor}
\usepackage{amsmath}
\usepackage{amssymb}
\usepackage{bm}
\usepackage{amsfonts} 
\begin{document}
\title{Lecture Notes 3}
\author{Minfei Chen}
\date{\today}
\maketitle
\section{最优策略(optimal policy)}
\subsection{定义}
一个策略 $\pi^*$ 是最优策略,如果$v_{\pi^*}(s) \ge v_{\pi}(s)$ 对于所有状态 $s$ 和其他所有策略 $\pi$.
\section{贝尔曼最优公式(BOE)}
\subsection{elementwise form}
$$
v(s) = \textcolor{blue}{\max}_{\pi} \sum_{a} \textcolor{blue}{\pi(a|s)} \left( \sum_{r} p(r|s, a)r + \gamma \sum_{s'} p(s'|s, a)v(s') \right), \quad \forall s \in \mathcal{S}
$$
\textcolor{blue}{\kaishu*已知:系统信息(状态转移概率、reword、$\gamma$);求解:最优策略$\pi$}
\subsection{最优化问题}
对于某个状态,最优的策略是选择action value最大的那一个action,即把它的概率设为1,其他的action的概率设为0。
$$
\pi(a|s) =
\begin{cases}
  1 & \text{if } a = a^* \\
  0 & \text{if } a \neq a^* 
\end{cases}
$$
$$
\text{where } a^* = \arg\max_{a} q(s, a).
$$
\subsection{压缩映射定理(contract mapping theorem)}
BOE可以写成:
$$
v = f(v)
$$
\textbf{不动点(fixed point)}:
$$x \in X \text{ is a fixed point of } f : X \to X \text{ if}$$
\[
f(x) = x
\]
\textbf{压缩映射(contract mapping)}:
$$\|f(x_1) - f(x_2)\| \le \gamma \|x_1 - x_2\|$$
$$\text{其中} \gamma \in (0, 1)$$
\textbf{压缩映射定理}:

对于任何形如 $x = f(x)$ 的方程,如果函数 $f$ 是一个\textbf{压缩映射 (contraction mapping)},那么:
\begin{itemize}
    \item \textbf{存在性 (Existence):} 存在一个不动点 $x^*$ 满足 $f(x^*) = x^*$。
    \item \textbf{唯一性 (Uniqueness):} 该不动点 $x^*$ 是唯一的。
    \item \textbf{算法/收敛性 (Algorithm):} 对于一个序列 $\{x_k\}$,如果 $x_{k+1} = f(x_k)$,那么当 $k \to \infty$ 时,序列 $x_k$ 将收敛到 $x^*$ ($x_k \to x^*$)。并且,该收敛是指数级快速的 (exponentially fast)。
\end{itemize}
\subsection{利用压缩映射定理求解BOE}
第一步:证明$f(v)$是一个contract mapping,公式中的$\gamma$恰好为discount rate

第二步:利用压缩映射定理来求出不动点$v^*$,即最高的状态值,并且可以知道解是唯一存在的,可以通过迭代的方法逐渐收敛到这个解
\subsection{BOE的最优性}
BOE实际上是一个特殊的贝尔曼公式$v^* = r_{\pi^*} + \gamma P_{\pi^*} v^*$,它所对应的策略是最优策略。

从状态值的角度看,假设 $v^*$ 是贝尔曼最优方程 $v = \max_{\pi} (r_{\pi} + \gamma P_{\pi} v)$ 的唯一解,
并且对于任何给定的策略 $\pi$,$v_\pi$ 是满足贝尔曼期望方程 $v_{\pi} = r_{\pi} + \gamma P_{\pi} v_{\pi}$ 的状态价值函数,那么:
\[
v^* \ge v_{\pi}, \quad \forall \pi
\]
贝尔曼最优公式所代表的其实是一个\textbf{贪心}最优策略,在这个策略下,每个状态都选择action value最大的action。
\section{最优策略分析}
影响策略的主要有三个因素:
\begin{itemize}
    \item 奖励设计 (Reward design): $r$
    \item 系统模型 (System model): $p(s'|s, a), p(r|s, a)$
    \item 折扣率 (Discount rate): $\gamma$
\end{itemize}
在系统模型固定的情况下,通过\textbf{调整reward和$\bm{\gamma}$},最优策略会发生很大的改变。

对于不同的$\gamma$,$\gamma$越大,agent会更注重长远的reward。$\gamma$越小,agent会更注重immediate reward。

对于不同的reward,调整reward的大小不一定会使最优策略发生改变。影响最优策略的是\textbf{reward之间的关系(relative reward)}。对奖励函数的线性修改不会改变action value的相对大小关系

{ % <--- 在开头添加左花括号,开始一个新的“组”
\kaishu
考虑一个马尔可夫决策过程,其最优状态价值函数为 $v^* \in \mathbb{R}^{|S|}$,满足贝尔曼最优方程 $v^* = \max_{\pi} (r_{\pi} + \gamma P_{\pi} v^*)$。

如果将每一个奖励 $r$ 都进行一次仿射变换 (affine transformation) 得到新的奖励 $ar+b$,其中 $a, b \in \mathbb{R}$ 且 $a > 0$,那么对应的新最优状态价值函数 $v'$ 也是 $v^*$ 的一个仿射变换:
\[
v' = av^* + \frac{b}{1-\gamma} \mathbf{1}
\]
其中 $\gamma \in (0, 1)$ 是折扣率,$\mathbf{1} = [1, \dots, 1]^T$ 是全1向量。

因此可以推断,\textbf{最优策略对于奖励信号的仿射变换是保持不变的}。
}

\textcolor{blue}{\kaishu*Discount rate实际上是对绕远路(meaningless detour)的一种惩罚。}
\end{document}

